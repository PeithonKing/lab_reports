\section{Abstract} 
	The phenomenon of splitting spectral lines in the presence of a magnetic field was first observed in 1896 by the Dutch physicist Pieter Zeeman as a broadening of the yellow D-lines of sodium in a flame held between strong magnetic poles. One of them is the splitting up of one spectral line into three components called the \textbf{`normal Zeeman effect'}. Here cadmium lamp is subjected to various magnetic field intensities and the amount of splitting is measured by the physical distance between the spectral lines. The red cadmium line (643.8nm) is studied using a fabry-perot interferometer. We can calculate a fairely precise value of Bohr magneton from the results.

\section{Aim}
	\begin{itemize}
		\item Using the Fabry-Perot interferometer, a telescope, a CMOS-camera and measurement software, the splitting up of the central line into two $\sigma$-lines is measured in wave numbers as a function of the magnetic flux density.
		\item From the results of point 1. a value for Bohr's magneton is evaluated.
		\item The light emitted within the direction of the magnetic field is qualitatively investigated.
	\end{itemize}