\section{Observations and Calculations}

	Least Count of the Oscilloscope Reading (for measuring $Q$ and $P$) is 0.2 cm. P (=maximum X deflection) = 7.8cm.

	\begin{table}[H]
    \centering
    \begin{tabular}{|c|c|c|c|c|c|c|}
        \hline
        frequency & c4   & r4      & c1     & r1      & $\epsilon$ & dissipation \\ \hline
        kHz       & pF   & k$\ohm$ & pF     & K$\ohm$ & ~          & factor      \\ \hline
        1         & 1150 & 1.10    & 366.67 & 3.45    & 737.76     & 0.0079      \\ \hline
        3         & 1050 & 1.00    & 333.33 & 3.15    & 670.69     & 0.0197      \\ \hline
        5         & 1000 & 0.98    & 326.67 & 3.00    & 657.28     & 0.0307      \\ \hline
        10        & 900  & 0.98    & 326.67 & 2.70    & 657.28     & 0.0553      \\ \hline
        15        & 800  & 0.96    & 320.00 & 2.40    & 643.86     & 0.0723      \\ \hline
        20        & 500  & 0.96    & 320.00 & 1.50    & 643.86     & 0.0602      \\ \hline
        25        & 300  & 0.94    & 313.33 & 0.90    & 630.45     & 0.0442      \\ \hline
        30        & 200  & 0.94    & 313.33 & 0.60    & 630.45     & 0.0354      \\ \hline
        35        & 150  & 0.92    & 306.67 & 0.45    & 617.03     & 0.0303      \\ \hline
        40        & 100  & 0.90    & 300.00 & 0.30    & 603.62     & 0.0226      \\ \hline
        50        & 100  & 0.90    & 300.00 & 0.30    & 603.62     & 0.0282      \\ \hline
    \end{tabular}
    \caption{observed capacitance value for $BaTiO_3$ at different frequency}
    \label{tab:1}
\end{table}

	\noindent From \hyperref[tab:data]{Table 1} we can see that:
	
	\begin{itemize}
		\item For $\nu_0 = 13.42MHz$ the value of slope $(m) = IQ = 0.233$.
		$$g = c\times\frac{13.42\times10^6}{0.233} = 1.944$$
		
		\item For $\nu_0 = 14.34MHz$ the value of slope $(m) = IQ = 0.239$.
		$$g = c\times\frac{14.34\times10^6}{0.239} = 2.025$$
		
		\item For $\nu_0 = 15.44MHz$ the value of slope $(m) = IQ = 0.239$.
		$$g = c\times\frac{15.44\times10^6}{0.253} = 2.060$$
	\end{itemize}

	\begin{center}\fbox{So, $g_{av} = 2.0098$}\end{center}	

	\begin{figure}[H]
		\centering
		\includegraphics[width=0.45\textwidth]{graph1.png}
		\caption{Q vs. $\sfrac{1}{I}$ for $\nu = 13.42MHz$}
		\label{graph:1}
	\end{figure}

	\begin{figure}[H]
		\centering
		\includegraphics[width=0.45\textwidth]{graph2.png}
		\caption{Q vs. $\sfrac{1}{I}$ for $\nu = 14.34MHz$}
		\label{graph:2}
	\end{figure}

	\begin{figure}[H]
		\centering
		\includegraphics[width=0.45\textwidth]{graph3.png}
		\caption{Q vs. $\sfrac{1}{I}$ for $\nu = 15.44MHz$}
		\label{graph:3}
	\end{figure}

	
