\section{Discussion}

\begin{itemize}
	\item \textbf{Absence of Equipment:} We completed the whole experiment in a single day. We were not provided with a measuring tape that day. So to measure the value of D (during measurement of the distance between the etalon) we used a normal 30cm metal scale. We can see that the value of D is more than 30cm. So we had to put the scale twice which can lead to a lot of errors. On top of that, it was a metal scale in an AC room (contracts and expands with temperature change). Fortunately, this error leads to decreased accuracy but does not affect the result's precision. To support this claim, let's calculate the $d$ for every data point wrt the intercept value obtained from \hyperref[graph:2]{Graph 2}.
	\begin{table}[h]
	\centering
	\resizebox{\columnwidth}{!}{%
		\begin{tabular}{|c|c|c|c|c|c|c|}
			\hline
			\textbf{\begin{tabular}[c]{@{}c@{}}Distance\\ (d) (cm)\end{tabular}} & \textbf{Count} & \textbf{\begin{tabular}[c]{@{}c@{}}Corrected\\ Count in 60s\end{tabular}} & \textbf{\begin{tabular}[c]{@{}c@{}}Net Count\\ Rate (R) \\ (per second)\end{tabular}} & \textbf{\begin{tabular}[c]{@{}c@{}}Product\\ $C=Rd^2$\end{tabular}} & \textbf{$\log(d)$} & \textbf{$\log(R)$} \\ \hline
			2.0                                                                  & 10462          & 9904                                                                      & 165.067                                                                               & 660.27                                                              & 0.693              & 5.106              \\ \hline
			2.5                                                                  & 7797           & 7239                                                                      & 120.650                                                                               & 754.06                                                              & 0.916              & 4.792              \\ \hline
			3.0                                                                  & 5819           & 5261                                                                      & 87.683                                                                                & 789.15                                                              & 1.098              & 4.473              \\ \hline
			3.5                                                                  & 4515           & 3957                                                                      & 65.950                                                                                & 807.89                                                              & 1.252              & 4.188              \\ \hline
			4.0                                                                  & 3624           & 3066                                                                      & 51.100                                                                                & 817.60                                                              & 1.386              & 3.933              \\ \hline
			4.5                                                                  & 2961           & 2403                                                                      & 40.050                                                                                & 811.01                                                              & 1.504              & 3.690              \\ \hline
			5.0                                                                  & 2424           & 1866                                                                      & 31.100                                                                                & 777.50                                                              & 1.609              & 3.437              \\ \hline
			5.5                                                                  & 2062           & 1504                                                                      & 25.067                                                                                & 758.27                                                              & 1.704              & 3.221              \\ \hline
			6.0                                                                  & 1824           & 1266                                                                      & 21.100                                                                                & 759.60                                                              & 1.791              & 3.049              \\ \hline
			7.0                                                                  & 1470           & 912                                                                       & 15.200                                                                                & 744.80                                                              & 1.945              & 2.721              \\ \hline
		\end{tabular}%
	}
	\caption{Inverse Square Law Data \& Calculations}
	\label{tab:3}
\end{table}
	We can clearly see that the standard deviation of the slopes is pretty low ($\approx 0.0009\%$) wrt to the slope. Hence, this error just results in reduced accuracy but does not affect the precision of the result.
	
	\item \textbf{Error in Observations:} We assumed the value of $m$ (number of fringes that appeared or disappeared) to be absolute while calculating the error. This is because uncertainty in calculating N can be solely due to random error. We cannot be sure about how much the value of $\Delta m$ can be. \textbf{Reported Error consists of only Systematic Error.} Other Sources of the error have not been calculated or estimated.

\end{itemize}