\section{Results \& Discussion}
	\begin{itemize}
		\item $A = 0.000109843 = 1.098\times10^{-4}\text{ per second}$
		\item $B = -0.00126324 = 1.263\times10^{-3}\text{ degree}$
		\item $Z_{Al} = 13.167$ with a percentage error of $1.28\%$.
		\item To enable active emission and deflection of the alpha particles and avoid their collision with air particles, vacuum must be produced.
		\item A scaling factor is represented by the proportionality factor (at logarithmic scale). The horizontal angular scale is being slightly displaced by the coefficient B. To better accurately define the points, I was unable to identify any additional explainable constant element in the equation.
		\item The small value of B shows that the readings are almost symmetric about the y axis, that is, the scattering rates are approximately equal for both positive and negative angles. The small value might be due to the fact that the detector is not perfectly aligned with the beam of alpha particles, or sampling error due to the limited number of readings.
		\item With the very thin gold foil, we noticed that the average reading at $+15^\circ$ and $-15^\circ$ are very far apart, although not that prominent for aluminium foil also we noticed the same phenomenon. The result was less prominent maybe due to the fact that Aluminium has a lower atomic number. \textit{Although, this is not observed for the thick foil}.
		\item To explain the above phenomenon, we should consider the fact, that when we used the thick foil on day 1, the source was perfectly perpendicular tot eh foil. So we got a perfectly symmetric graph. But when we returned the next day to use the thin foil, the source might be slighlty tilted while changing the foil. This might have caused the asymmetry in the readings.
	\end{itemize}