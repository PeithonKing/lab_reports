\section{Observations and Calculations}
	\subsection{Recording the scattering rate\\as function of the angle}

		Our experimentally observed data for measuring the counts and hence the decay rate for different angles is shown in \hyperref[tab:1]{Table 1}. The graph for the same is shown in \hyperref[graph:1]{Figure 4}.

		\begin{table}[H]
    \centering
    \begin{tabular}{|c|c|c|c|c|c|c|}
        \hline
        frequency & c4   & r4      & c1     & r1      & $\epsilon$ & dissipation \\ \hline
        kHz       & pF   & k$\ohm$ & pF     & K$\ohm$ & ~          & factor      \\ \hline
        1         & 1150 & 1.10    & 366.67 & 3.45    & 737.76     & 0.0079      \\ \hline
        3         & 1050 & 1.00    & 333.33 & 3.15    & 670.69     & 0.0197      \\ \hline
        5         & 1000 & 0.98    & 326.67 & 3.00    & 657.28     & 0.0307      \\ \hline
        10        & 900  & 0.98    & 326.67 & 2.70    & 657.28     & 0.0553      \\ \hline
        15        & 800  & 0.96    & 320.00 & 2.40    & 643.86     & 0.0723      \\ \hline
        20        & 500  & 0.96    & 320.00 & 1.50    & 643.86     & 0.0602      \\ \hline
        25        & 300  & 0.94    & 313.33 & 0.90    & 630.45     & 0.0442      \\ \hline
        30        & 200  & 0.94    & 313.33 & 0.60    & 630.45     & 0.0354      \\ \hline
        35        & 150  & 0.92    & 306.67 & 0.45    & 617.03     & 0.0303      \\ \hline
        40        & 100  & 0.90    & 300.00 & 0.30    & 603.62     & 0.0226      \\ \hline
        50        & 100  & 0.90    & 300.00 & 0.30    & 603.62     & 0.0282      \\ \hline
    \end{tabular}
    \caption{observed capacitance value for $BaTiO_3$ at different frequency}
    \label{tab:1}
\end{table}
		\begin{figure}[h]
			\centering
			\label{graph:1}
			\includegraphics[width=0.8\columnwidth]{images/graph1.png}
			\caption{N vs $\theta$ graph for \hyperref[tab:1]{Table 1}}
		\end{figure}

		In \hyperref[graph:1]{Figure 4} we have fitted the curve with the function:
		$$y = \frac{A}{\sin^4\frac{x-B}{2}}$$

		where $A$ and $B$ are the constants to be determined, $x$ is the angle $(\theta)$ and $y$ is the counts per second.

		From \hyperref[graph:1]{the graph} we got the values of the constants as:

		$$A = 0.000109843 = 1.098\times10^{-4}\text{ per second}$$
		$$B = -0.00126324 = 1.263\times10^{-3}\text{ degree}$$
	
	\subsection{Determining the atomic number of Al}

		Our experimentally observed data for measuring the counts and hence the decay rate for different materials keeping all other conditions constant is shown in \hyperref[tab:2]{Table 2}.

		% \begin{table}[H]
% 	\centering
% 	\resizebox{0.5\columnwidth}{!}{%
% 	\begin{tabular}{|c|c|}
% 	\hline
% 	\begin{tabular}[c]{@{}c@{}}Accelerating\\ Voltage $(U_A)\;V$\end{tabular} & \begin{tabular}[c]{@{}c@{}}Collector\\ Current $(I_E)\;nA$\end{tabular} \\ \hline
% 	0 & 0 \\ \hline
% 	7.5 & 0 \\ \hline
% 	8 & 1 \\ \hline
% 	8.5 & 3 \\ \hline
% 	9 & 4 \\ \hline
% 	9.5 & 5 \\ \hline
% 	10 & 5 \\ \hline
% 	10.5 & 6 \\ \hline
% 	11 & 7 \\ \hline
% 	11.5 & 8 \\ \hline
% 	12 & 8 \\ \hline
% 	12.5 & 9 \\ \hline
% 	13 & 9 \\ \hline
% 	14 & 9 \\ \hline
% 	15 & 10 \\ \hline
% 	16 & 11 \\ \hline
% 	17 & 11 \\ \hline
% 	17.5 & 12 \\ \hline
% 	18 & 11 \\ \hline
% 	18.5 & 10 \\ \hline
% 	19 & 9 \\ \hline
% 	20 & 7 \\ \hline
% 	21 & 1 \\ \hline
% 	22 & 0 \\ \hline
% 	23 & -3 \\ \hline
% 	24 & -3 \\ \hline
% 	25 & 0 \\ \hline
% 	26 & 4 \\ \hline
% 	28 & 12 \\ \hline
% 	29 & 15 \\ \hline
% 	30 & 20 \\ \hline
% 	32 & 26 \\ \hline
% 	34 & 32 \\ \hline
% 	35 & 33 \\ \hline
% 	36 & 32 \\ \hline
% 	38 & 23 \\ \hline
% 	39 & 18 \\ \hline
% 	40 & 12 \\ \hline
% 	41 & 6 \\ \hline
% 	43 & 3 \\ \hline
% 	44 & 4 \\ \hline
% 	45 & 9 \\ \hline
% 	46 & 15 \\ \hline
% 	47 & 24 \\ \hline
% 	48 & 35 \\ \hline
% 	49 & 48 \\ \hline
% 	50 & 55 \\ \hline
% 	52 & 76 \\ \hline
% 	54 & 88 \\ \hline
% 	54.5 & 90 \\ \hline
% 	55 & 91 \\ \hline
% 	55.5 & 91 \\ \hline
% 	56 & 90 \\ \hline
% 	56.5 & 88 \\ \hline
% 	57 & 88 \\ \hline
% 	58 & 85 \\ \hline
% 	59 & 82 \\ \hline
% 	60 & 81 \\ \hline
% 	62 & 85 \\ \hline
% 	64 & 100 \\ \hline
% 	65 & 109 \\ \hline
% 	66 & 118 \\ \hline
% 	68 & 152 \\ \hline
% 	70 & 181 \\ \hline
% 	\end{tabular}%
% 	}
% 	\caption{Dataset 2}
% 	\label{tab:2}
% \end{table}





\begin{table}[H]
	\centering
	\resizebox{\columnwidth}{!}{%
	\begin{tabular}{|c|c|c|c|c|c|c|}
	\cline{1-3} \cline{5-7}
	Sl. No. & \begin{tabular}[c]{@{}c@{}}Accelerating\\ Voltage $(U_A)\;V$\end{tabular} & \begin{tabular}[c]{@{}c@{}}Collector\\ Current $(I_E)\;nA$\end{tabular} &  & Sl. No. & \begin{tabular}[c]{@{}c@{}}Accelerating\\ Voltage $(U_A)\;V$\end{tabular} & \begin{tabular}[c]{@{}c@{}}Collector\\ Current $(I_E)\;nA$\end{tabular} \\ \cline{1-3} \cline{5-7} 
	1 & 0 & 0 &  & 33 & 34 & 32 \\ \cline{1-3} \cline{5-7} 
	2 & 7.5 & 0 &  & 34 & 35 & 33 \\ \cline{1-3} \cline{5-7} 
	3 & 8 & 1 &  & 35 & 36 & 32 \\ \cline{1-3} \cline{5-7} 
	4 & 8.5 & 3 &  & 36 & 38 & 23 \\ \cline{1-3} \cline{5-7} 
	5 & 9 & 4 &  & 37 & 39 & 18 \\ \cline{1-3} \cline{5-7} 
	6 & 9.5 & 5 &  & 38 & 40 & 12 \\ \cline{1-3} \cline{5-7} 
	7 & 10 & 5 &  & 39 & 41 & 6 \\ \cline{1-3} \cline{5-7} 
	8 & 10.5 & 6 &  & 40 & 43 & 3 \\ \cline{1-3} \cline{5-7} 
	9 & 11 & 7 &  & 41 & 44 & 4 \\ \cline{1-3} \cline{5-7} 
	10 & 11.5 & 8 &  & 42 & 45 & 9 \\ \cline{1-3} \cline{5-7} 
	11 & 12 & 8 &  & 43 & 46 & 15 \\ \cline{1-3} \cline{5-7} 
	12 & 12.5 & 9 &  & 44 & 47 & 24 \\ \cline{1-3} \cline{5-7} 
	13 & 13 & 9 &  & 45 & 48 & 35 \\ \cline{1-3} \cline{5-7} 
	14 & 14 & 9 &  & 46 & 49 & 48 \\ \cline{1-3} \cline{5-7} 
	15 & 15 & 10 &  & 47 & 50 & 55 \\ \cline{1-3} \cline{5-7} 
	16 & 16 & 11 &  & 48 & 52 & 76 \\ \cline{1-3} \cline{5-7} 
	17 & 17 & 11 &  & 49 & 54 & 88 \\ \cline{1-3} \cline{5-7} 
	18 & 17.5 & 12 &  & 50 & 54.5 & 90 \\ \cline{1-3} \cline{5-7} 
	19 & 18 & 11 &  & 51 & 55 & 91 \\ \cline{1-3} \cline{5-7} 
	20 & 18.5 & 10 &  & 52 & 55.5 & 91 \\ \cline{1-3} \cline{5-7} 
	21 & 19 & 9 &  & 53 & 56 & 90 \\ \cline{1-3} \cline{5-7} 
	22 & 20 & 7 &  & 54 & 56.5 & 88 \\ \cline{1-3} \cline{5-7} 
	23 & 21 & 1 &  & 55 & 57 & 88 \\ \cline{1-3} \cline{5-7} 
	24 & 22 & 0 &  & 56 & 58 & 85 \\ \cline{1-3} \cline{5-7} 
	25 & 23 & -3 &  & 57 & 59 & 82 \\ \cline{1-3} \cline{5-7} 
	26 & 24 & -3 &  & 58 & 60 & 81 \\ \cline{1-3} \cline{5-7} 
	27 & 25 & 0 &  & 59 & 62 & 85 \\ \cline{1-3} \cline{5-7} 
	28 & 26 & 4 &  & 60 & 64 & 100 \\ \cline{1-3} \cline{5-7} 
	29 & 28 & 12 &  & 61 & 65 & 109 \\ \cline{1-3} \cline{5-7} 
	30 & 29 & 15 &  & 62 & 66 & 118 \\ \cline{1-3} \cline{5-7} 
	31 & 30 & 20 &  & 63 & 68 & 152 \\ \cline{1-3} \cline{5-7} 
	32 & 32 & 26 &  & 64 & 70 & 181 \\ \cline{1-3} \cline{5-7} 
	\end{tabular}%
	}
	\caption{dataset II}
	\label{tab:2}
	\end{table}

		Now, we know that:
		\begin{itemize}
			\item $Z_{Au}$ = 79
			\item We used equal thickness foils. So, $d_{Au}$ = $d_{Al}$.
			\item $N_{Au} = 0.48cps$ 
			\item $N_{Al} = 0.01333cps$.
		\end{itemize}

		from \hyperref[eq:2]{Equation 2} we get the value of $Z_{Al}$ as:

		$$Z_{Al} = 79\sqrt{\frac{0.01333}{0.48}}$$

		$$\fbox{Z = 13.167}$$
