\section{Conclusion}
	\begin{itemize}
		\item We conducted an experiment on Compton scattering of electrons, which involves the scattering of X-rays by free electrons.

		\item The observed data in our experiment shows that the intensity of scattered photons decreases as the scattering angle increases. This observation agrees with the Compton formula, which describes the energy transfer from photons to electrons in the scattering process.

		\item We also found that the energy of scattered photons decreases as the scattering angle increases, which again agrees with the Compton formula. This is because the energy of scattered photons is related to the energy lost by the electrons in the scattering process.

		\item Our experiment also revealed that the detector used in the experiment had different calibration factors for different materials. The calibration factor for brass was found to be approximately two times that of aluminum. This implies that for particular cross-sections, the relative intensity for brass is more than for aluminum.

		\item We calculated the rest mass of electrons in our experiment, but the result had a significant error. We think that this error can be minimized by taking precautions with the lead box used in the experiment. This suggests that our experiment can be improved by using better shielding to reduce unwanted background radiation and improve the accuracy of our measurements.

		\item Our experiment demonstrates that the Compton scattering process can be used to determine the rest mass of charged particles like electrons. By measuring the scattered photon energy and angle, along with the incident photon energy, it is possible to calculate the rest mass of the electron. This technique is widely used in experimental physics to measure the masses of particles.
	\end{itemize}
