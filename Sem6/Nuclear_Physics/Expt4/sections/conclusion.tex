\section{Results and Discussion}
	\begin{itemize}
		\item We find that the coincidence rate is maximum for $185^\circ$ and minimum for $90^\circ$. Accoriding to the theory, the maximum count should have been at $180^\circ$. This must have been due to the random errors in the experiment. To reduce this, we could have taken the data for a longer time period.
		\item Since the experiment took a long time, we couldn't take more data points at different angles. A data point at all angles  $10^\circ$ apart from $90^\circ$ to $270^\circ$ would have given a better picture of the coincidence rate.
	\end{itemize}
	\subsection{Errors in the experiment:}
	Here some of the sources of errors in the experiment:
	\begin{itemize}
		\item Small sample time.
		\item incorrect detector and radioactive sample alignment. The experimenter was mostly allowed to use his or her judgement when establishing the angle because there wasn't a formal setup in place.
		\item Systematic inaccuracy brought on by a goniometer flaw that prevents it from accurately determining the angle.
	\end{itemize}
	\subsection{Conclusion}
		Hence we have studied Gamma-Gamma coincidence and verified the fact the coincidence rates decreases for angles beyond the $180^\circ$. The presence of significant coincidences beyond this angle is because the velocity of the positronium is not not zero in the lab frame. We have also studied the process of decay of $Na^22$. Some precautions should be taken while doing the experiment.
		\begin{itemize}
			\item The sample shouldn't be touched, even though it's in a case. After setting the appartus, it should not be disturbed.
			\item The histogram should be saved before starting the experiment.
			\item The coincidences should be measured in a longer time period.
			\item The alignment of the sample should be correct. In case of $90^\circ$ angle with the detector, the sample should be placed slanted to allow the detections to happen.
		\end{itemize}