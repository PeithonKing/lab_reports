\section{Conclusion}

	Suspected Sources of errors:
	\begin{enumerate}
		\item \textbf{Inaccurate measurement of groove spacing:} The discrepancy between the assumed groove spacing of 5 mm and the measured groove spacing of almost 7 mm could have introduced errors in the distance measurements for the inverse square law part of the experiment. This discrepancy could result in inaccurate values for the distances between the source and the detector, leading to potential deviations from the true inverse square relationship between radiation intensity and distance.
		\item \textbf{Unintended variations in source orientation:} The use of disk-shaped radioactive sources without knowing the "right" or "wrong" side could introduce inconsistencies in the data. The two flat surfaces of the disk had different emission characteristics. Inadvertently using the "wrong" side of the disk for measurements could result in inaccurate data. This could lead to potential discrepancies in the observed counts and the actual emissions from the source, affecting the accuracy of the results.
		\item \textbf{Relatively small sample size for counting statistics:} The use of a small sample size of 100 data points for the counting statistics part of the experiment may not fully represent the underlying distribution of counts. This could result in a less accurate estimation of the true Gaussian distribution and may affect the interpretation of the results. A larger sample size, such as 1000 data points as suggested by your Monte Carlo simulation, could have provided a better approximation to the expected Gaussian distribution.
		% \item \textbf{Random errors during the experiment:} Random errors could have occurred during the experiment, such as uncertainties in voltage readings, positioning of the detector, or fluctuations in environmental conditions, which could have introduced variability in the measurements. These random errors may not have been accounted for and could have influenced the accuracy and precision of the results.
	\end{enumerate}

	In this experiment, we confirmed the inverse square law for radiation intensity using a GM counter and observed a clear correlation between counts and distance from the source at a fixed operating voltage. We determined the efficiency of the GM counter by comparing expected and measured counts, providing valuable information for future experiments. Identified sources of error, such as inaccurate groove spacing measurement on the sample holder, highlighted the importance of precise measurements. The observed counts followed a Gaussian distribution around a mean value, underscoring the need for statistical analysis in interpreting experimental data.