\section{Conclusion}
	The experimental results for the endpoint energies and ranges of $Tl^{204}$ and $Sr^{90}$ were consistent with the theoretical values, although there was a minor difference. The count rate versus thickness graph obtained for the samples followed an exponential decay pattern, allowing us to calculate the half-thickness value.

	We also obtained the expected graph for backscattering, with an initial increase in count rate that eventually reaches saturation.

	By studying the attenuation of Bremsstrahlung radiation by $Cu$, $Al$, and Perspex, we found that more attenuation occurs when Perspex is facing the source. Conversely, we observed higher Bremsstrahlung counts when metals are facing the source. Among the metals, $Cu$ gave more counts than $Al$, in line with the theory that the amount of Bremsstrahlung increases with the atomic number of the absorbing material.

	To determine the half-life of the Cs/Ba$^{137m}$ isotope generator, we recorded a 5-minute video of the GM counter and measured the counts every 10 seconds. Using the first data point as $N_0$, we obtained a graph of log(counts) versus time that followed a straight line with a negative slope, consistent with the theory.

	There are various factors that might have contributed to the minor differences in values obtained and shapes of graphs plotted during the experiment. Some possible sources of errors are:
	\begin{itemize}
		\item The background radiation detected by the GM counter, which may lead to an overestimation of the sample's activity.
		\item The absorption of radiation in the air between the sample and the GM counter.
		\item The beta particles from the sample might have been absorbed by the GM tube's window.
		\item The "dead time" of the GM counter, which is the time during which it cannot detect another event, might have led to a reduction in the count rate and an overestimation of the sample's activity.
		\item Electronic noise produced by the components of the GM counter might have led to an increase in the count rate and an overestimation of the sample's activity.
		\item Fluctuations in the GM counter's operating voltage might have led to changes in the count rate and an overestimation or underestimation of the sample's activity.
	\end{itemize}