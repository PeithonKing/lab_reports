\begin{abstract}
	In this experiment we will study the various solid properties of Bismuth like Magnetoresistance and Hall Effect.
	\begin{description}
		\item[Magnetoresistance] When the magnetic field is turned on, the resistance of a sample varies. Magnetoresistance is a material's capacity to alter the value of its electrical resistance when subjected to an external magnetic field. The amplitude of the impact is fairly small ($\approx1\%$) at normal temperature, but increases to roughly 50\% at low temperatures in gigantic magneto resistive multilayer systems. In certain perovskite systems, effects of more than 95\% change in resistivity have recently been discovered.
		\item[Hall Effect] The Hall effect is the phenomenon of appearance of a potensial difference perpendicular to the direction of curent flow if a perpendicular magnetic field is applied. In the operating region, the Hall effect is a linear effect, which means that the voltage is proportional to the current and the magnetic field. The Hall coefficient is a measure of the Hall effect. It is defined as the ratio of the Hall voltage to the product of the current and the magnetic field. The Hall coefficient is a material property and is independent of the geometry of the sample.
	\end{description}
\end{abstract}