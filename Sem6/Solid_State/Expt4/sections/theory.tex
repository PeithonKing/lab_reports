\section{Theory}
    \subsection{Magnetoresistance}
		Under the influence of a magnetic field, the resistance of some materials change significantly. This effect is popularly known as magnetoresistance of the material. This effect can be observed due to the fact that the drift velocity of the carriers is not same. In the presence of the Magnetic field, the carriers drift in the direction of the field. In this condition, the hall voltage compensates the lorentz force for carriers with average velocity. The slower carriers are overcompensated and the faster carriers are undercompensated. This disturbs the flow of electrons along the direction of flow of current; hence reducing the mean free path and increasing the resistance of the material. In this condition, the hall voltage is given by the formula:
		
		$$V = E_yt = |v\times H|$$

		where $E_y$ and $H$ are the electric field and the magnetic fields, $t$ is the thickness of the sample, and $v$ is the drift velocity of the carriers.

		