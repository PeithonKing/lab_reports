\section{Theory}
    \subsection{Magnetoresistance}
		Under the influence of a magnetic field, the resistance of some materials change significantly. This effect is popularly known as magnetoresistance of the material. This effect can be observed due to the fact that the drift velocity of the carriers is not same. In the presence of the Magnetic field, the carriers drift in the direction of the field. In this condition, the hall voltage compensates the lorentz force for carriers with average velocity. The slower carriers are overcompensated and the faster carriers are undercompensated. This disturbs the flow of electrons along the direction of flow of current; hence reducing the mean free path and increasing the resistance of the material. In this condition, the hall voltage is given by the formula:
		
		$$V = E_yt = |v\times H|$$

		where $E_y$ and $H$ are the electric field and the magnetic fields, $t$ is the thickness of the sample, and $v$ is the drift velocity of the carriers.

		The change in resistivity $\Delta\rho$ is positive for both magnetic field parallel $\Delta \rho_\parallel$ and transverse $\Delta \rho_T$ to the current direction with $\rho_T>\rho_\parallel$. There are three different kinds of magnetoresistance, depending on the structure of the electron orbitals at the Fermi surface:

		\begin{enumerate}
			\item The magnetic field has the effect of raising the cyclotron frequency of the electron in its confined orbit in metals with closed Fermi surfaces where the electrons are restricted to their orbit in k-space.
			\item The magnetoresistance for metals with an equal number of electrons and holes rises with magnetic field up to the highest observed fields and is unaffected by crystallographic orientation. These materials include bismuth.
			\item In some crystallographic orientations, Fermi surfaces with open orbits will show significant magnetoresistance for applied fields, but the resistance will saturate in other crystallographic directions where the orbits are closed.
		\end{enumerate}

	\subsection{Hall Effect}
		This effect was discovered by Edwin Hall in 1879. This is the production of a voltage across a conductor when an electric current is passed through it in the presence of a magnetic field. This results in the separation of charge carriers into two regions, one with positive charge and the other with negative charge. This will result in an electric field perpendicular to both the current and the magnetic field (in the $\vec{H}\times\vec{I}$ direction). 

		\begin{equation}
			\vec{E_h} = R \vec{J}\times\vec{H}
			\label{eqn:1}
		\end{equation}
		
		where $E_h$ and $\vec{H}$ are the electric field and the magnetic fields, $\vec{J}$ is the current density and the proportionality constant $R$ is called the \textbf{Hall Coefficient}.

		Now, let us consider a bar of a semiconductor, having dimensions, x, y and z. Let $\vec{J}$ be directed along x and $\vec{H}$ along z, then  $E_h$ will be along y. Then we could write:

		\begin{equation}
			R = \frac{V_h/y}{JH} = \frac{V_h\times z}{IH}
			\label{eqn:2}
		\end{equation}