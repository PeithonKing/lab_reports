\section{Theory}
    \subsection{Magnetoresistance}
		Under the influence of a magnetic field, the resistance of some materials change significantly. This effect is popularly known as magnetoresistance of the material. This effect can be observed due to the fact that the drift velocity of the carriers is not same. In the presence of the Magnetic field, the carriers drift in the direction of the field. In this condition, the hall voltage compensates the lorentz force for carriers with average velocity. The slower carriers are overcompensated and the faster carriers are undercompensated. This disturbs the flow of electrons along the direction of flow of current; hence reducing the mean free path and increasing the resistance of the material. In this condition, the hall voltage is given by the formula:
		
		$$V = E_yt = |v\times H|$$

		where $E_y$ and $H$ are the electric field and the magnetic fields, $t$ is the thickness of the sample, and $v$ is the drift velocity of the carriers.

		The change in resistivity $\Delta\rho$ is positive for both magnetic field parallel $\Delta \rho_\parallel$ and transverse $\Delta \rho_T$ to the current direction with $\rho_T>\rho_\parallel$. There are three different kinds of magnetoresistance, depending on the structure of the electron orbitals at the Fermi surface:

		\begin{enumerate}
			\item The magnetic field has the effect of raising the cyclotron frequency of the electron in its confined orbit in metals with closed Fermi surfaces where the electrons are restricted to their orbit in k-space.
			\item The magnetoresistance for metals with an equal number of electrons and holes rises with magnetic field up to the highest observed fields and is unaffected by crystallographic orientation. These materials include bismuth.
			\item In some crystallographic orientations, Fermi surfaces with open orbits will show significant magnetoresistance for applied fields, but the resistance will saturate in other crystallographic directions where the orbits are closed.
		\end{enumerate}

	\subsection{Hall Effect}
		The Hall effect is the production of a voltage across a conductor when an electric current is passed through it. The Hall voltage is proportional to the current density and the magnetic field perpendicular to the current. The Hall voltage is given by the formula:
		
		$$V = E_yt = |v\times H|$$

		where $E_y$ and $H$ are the electric field and the magnetic fields, $t$ is the thickness of the sample, and $v$ is the drift velocity of the carriers.

		The Hall effect is a direct consequence of the Lorentz force on the charge carriers in a magnetic field. The Hall effect is a linear effect, meaning that the Hall voltage is proportional to the current density and the magnetic field perpendicular to the current. The Hall coefficient is the ratio of the Hall voltage to the product of the current density and the magnetic field perpendicular to the current. The Hall coefficient is a material property and is independent of the applied current density and the magnetic field. The Hall coefficient is a measure of the mobility of the charge carriers in the material. The Hall coefficient is given by the formula:
		
		$$R_H = \frac{V}{I\times B}$$

		where $V$ is the Hall voltage, $I$ is the current density, and $B$ is the magnetic field perpendicular to the current.

		The Hall effect is a linear effect, meaning that the Hall voltage is proportional to the current density and the magnetic field perpendicular to the current. The Hall coefficient is the ratio of the Hall voltage to the product of the current density and the magnetic field perpendicular to the current. The Hall coefficient is a material property and is independent of the applied current density and the magnetic field. The Hall coefficient is a measure of the mobility of the charge carriers in the material. The Hall coefficient is given by the formula:
		
		$$R_H = \frac{V}{I\times B}$$

		where $V$ is the Hall voltage, $I$ is the current density, and $B$ is the magnetic field perpendicular to the current.

		The Hall effect is a linear effect, meaning that the Hall voltage is proportional to the current density and the magnetic field perpendicular to the current. The Hall coefficient is the ratio of the Hall voltage to the product of the current density and the magnetic field perpendicular to the current. The Hall coefficient is a material property and is independent of the applied current density and the magnetic field. The Hall coefficient is a measure of the mobility of the charge carriers in the material. The Hall coefficient is given by the formula: