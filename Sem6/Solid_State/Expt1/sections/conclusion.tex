\section{Conclusion}
	Our experiments utilizing the four probe method allowed us to accurately determine the resistivities of various samples, including both semiconductors and metals. Additionally, we were able to calculate the band gap of a semiconductor sample through temperature-dependent measurements. We observed that the resistivity of semiconductors decreased with an increase in temperature, in line with theoretical predictions.

	However, it's important to note that there were some propagational errors in our experiments, which could be attributed to fluctuations in supply voltage, carrier injection, or impurities in the sample material. We also made assumptions about the uniformity of resistivity in our samples, which may not always hold true. Nevertheless, these errors were within the acceptable error limits, except for a few exceptions that were previously discussed.

	The four probe method has emerged as a widely used technique for measuring resistance and resistivity of thin wafers composed of different materials. It has proven to be superior to the conventional two-probe method in terms of accuracy and reliability in our experiments.