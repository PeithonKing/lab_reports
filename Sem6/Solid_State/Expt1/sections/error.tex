\section{Error Analysis}
	\subsection{Estimation of Error}
		While calculating the resistivity of materials, we didn't measure the thickness of the sample. So we assumed that the error in the thickness of the sample is negligible. Thus the only quantity that contributes to the error is the slope of the graph. Thus:

		$$\Delta\rho = \rho\times\frac{\Delta slope}{slope}$$

		Thus the corrected values of resistivity are:
		\begin{itemize}
			\item $\rho_{Ge} = 19.93\pm0.011\;\ohm cm$
			\item $\rho_{Al} = 2.91\times10^{-6}\pm1.62\times10^{-8}\;\ohm cm$
			\item $\rho_{Si} = 23.8\pm0.161\;\ohm cm$
		\end{itemize}

		Also, in the calculation of the band gap, the only quantity that contributes to the error is the slope of the graph. Thus:

		$$\Delta E_g = E_g\times\frac{\Delta slope}{slope}$$

		Thus the corrected value of band gap is:

		$$E_g = 0.703\pm0.0074eV$$

	\subsection{Suspected Sources of Error}
		\begin{enumerate}
			\item The sample should be of uniform thickness.
			\item The Aluminium was commercial grade aluminium, so it was not pure. So the resistivity of the sample was not accurate.
			\item The Four probe method works under the assumption that the sample used is semi infinite, which may not be accurate in every situation.
			\item Instability in the data due to improper contact.
			\item Variation of doping in the sample.
		\end{enumerate}
		
	\subsection{Precautions to avoid error}
		\begin{enumerate}
			\item When the zero adjustment cannot be done, the offset is noted and reduced from the data to provide corrected values.
			\item To avoid loose contacts, the springs are tightened on the four probes appropriately.
			\item The sample is cleaned before measurements and appropriate ranges are used in digital measurement devices.
			\item Sufficient time was given for the temperature to stabilize before noting the measurments.
		\end{enumerate}