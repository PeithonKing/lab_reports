\section{Error analysis}
	In $IC 555$ multivibrator circuit the observed value of frequency and duty cycle is different from the theoretical value. For the circuit with a duty cycle less than or equal to $50\%$.

	There may be several reasons for the high error in the observed values of frequency and duty cycle in the circuit using IC555. One possible reason is the difference between the actual values of resistors and capacitors used in the circuit and the ideal values assumed during calculations, which can affect the frequency and duty cycle. Another factor could be a loose connection or instability in the circuit, resulting in variations in the observed values.

	$$\delta f=33.8\%$$

	$$\delta (\text{duty cycle}) = 25.3\%$$

	For a circuit with a duty cycle of more than $50\%$.

	$$\delta f=2.05\%$$

	$$\delta (\text{duty cycle}) = 9.7\%$$

	Similarly, the calculated magnetic moment may also have errors. Other forces such as air resistance and Lenz force, which are neglected in the calculations, could contribute to the error. Additionally, inaccuracies in the measured values of resistance (R) and distance (z) used in the calculations can also lead to errors in the magnetic moment value.
	
	In summary, the high error in the observed values of frequency, duty cycle, and magnetic moment may be attributed to various factors, including differences in actual component values, circuit instability, neglected forces, and inaccuracies in measured values.