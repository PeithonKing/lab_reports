\section{Conclusion}
	In summary, we have accomplished the following in our study of UV-Vis spectroscopy for optical bandgap measurement:

	\begin{itemize}
		\item We were able to determine the direct and indirect bandgaps for different materials by measuring their light absorption spectra as a function of wavelength. This method provides valuable information on the electronic transitions in semiconductors.
		\item We have found that the bandgap value obtained from UV-Vis spectroscopy is slightly lower than the actual bandgap value of the material, which can be attributed to impurities or other factors affecting the sample.
		\item We have also observed that indirect bandgap materials exhibit a more continuous and wider range of absorbance compared to direct bandgap materials, which makes them better suited for use in solar cells.
		\item To minimize errors in bandgap calculations, we recommend using pure materials and a vacuum chamber to avoid contaminants and ensure consistent material and light interaction.
	\end{itemize}

	In conclusion, UV-Vis spectroscopy is a valuable tool for studying the electronic properties of semiconductors and can provide insights into their potential applications in various fields, such as solar cell technology. However, careful consideration of the factors that can affect the accuracy of the results is necessary to obtain reliable data.