\section{Error}

	When measuring the band gap of materials, the obtained value is always smaller than the real band gap value. For instance, the real band gap for CdS is $2.42~\text{eV}$ and for ZnTe is $2.26~\text{eV}$. However, the measured values for CdS and ZnTe are $2.31~\text{eV}$ and $2.03~\text{eV}$, respectively. Therefore, there is a certain amount of error in the measured values.

	The error in $E_g$ can be calculated as follows:
	\begin{itemize}
	\item $\delta E_g(\text{CdS})=4.5\%$
	\item $\delta E_g(\text{ZnTe})=10.17\%$
	\end{itemize}

	Unfortunately, it is not possible to calculate the error value for the polymer because the real bandgap value is unknown. Several factors can contribute to errors in bandgap calculations, including the presence of contaminants in the sample, uneven sample thickness, or errors in background data. Additionally, the second slope in the polymer example is not very useful for bandgap calculations, which may introduce inaccuracies in the calculated value. To minimize random errors, we conducted our experiment in a vacuum chamber using only pure light.

