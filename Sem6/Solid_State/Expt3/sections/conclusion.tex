\section{Conclusion}

	In conclusion, our experiments for measuring the Hall coefficient have shown that the polarity of the Hall voltage is opposite for p-type and n-type samples, resulting in a change in the sign of $R_H$. We found that the calculated absolute value of $R_H$ for n-type germanium is higher than that of p-type germanium, indicating a lower charge density in the former. This is because the absolute value of $R_H$ is inversely proportional to the charge density, which is in turn proportional to the number of charge carriers.

	Moreover, we observed that the graph of $V_H$ vs $I_P$ is not linear for both p-type and n-type samples at high probe current, which can be attributed to the heating of the sample. Our temperature-dependent measurements revealed that the value of $R_H$ decreases with increasing temperature for the p-type sample and eventually becomes negative, as expected due to an increase in the number of negative charge carriers with an increase in temperature. The decrease is not linear because of the significant difference between the mobilities of the two types of charge carriers at low temperatures.
