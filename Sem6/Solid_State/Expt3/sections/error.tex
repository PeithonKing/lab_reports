\section{Error}
	For $R_H $ value calculation at a constant temperature:

	$$\frac{\delta R_H}{R_H}=\sqrt{
	\left(\frac{\delta slope}{slope}\right)^2+\left(\frac{\delta B}{B}\right)^2}$$

	For Ge n type $\delta R_H=0.041 cm^3 coulomb^{-1}$

	For Ge p type $\delta R_H= 0.017 cm^3 coulomb^{-1}$

	The error in the measured value of $R_H$ was found to be small and can be considered negligible. However, the observed variation in the obtained value of $R_H$ could be attributed to several factors, including:

	\begin{itemize}
		\item Fluctuations in the coil current during the experiment, which may result in variations in the magnetic field strength experienced by the sample. Such fluctuations could be caused by external disturbances or instabilities in the power supply.
		\item Variations in the ambient temperature of the room, which could affect the electrical properties of the sample and lead to variations in the measured value of $R_H$. Such variations could be due to fluctuations in air conditioning or heating systems, or other environmental factors.
		\item Thermal effects on the sample, caused by high probe current or non-zero thermal EMF. These effects could cause changes in the temperature of the sample, leading to variations in the measured value of $R_H$. The magnitude of such effects may depend on the specific characteristics of the sample and the experimental setup.
		\item Impurities in the sample, which could introduce additional sources of scattering and lead to variations in the measured value of $R_H$. Such impurities may be present in the bulk of the material or at the interfaces between different layers or regions of the sample.
	\end{itemize}

	It is important to note that these factors may not be the only sources of error in the experiment, and that other uncontrolled variables could also contribute to the observed variations in the measured value of $R_H$. Therefore, a careful analysis of the experimental conditions and possible sources of error is crucial for the interpretation of the results and the assessment of their reliability.