\section{Results \& Discussions}
	\begin{enumerate}
		\item By analyzing graphs 6 and 7, we can observe that the values for the parameters, VOC, ISC, and $P_{max}$ vary depending on the type of filter used.

		\item The amount of energy and intensity of the incident radiation affects the potential across a solar cell, as it influences the generation of high-energy charge carriers and their migration across the cell.

		\item The energy of the incident radiation determines the kinetic energy of excited electrons, while the intensity determines the number of charge carriers released.

		\item The energy of a photon is dependent on its wavelength, while the intensity is determined by the amount of transmitted light after passing through a filter.

		\item The reason why the VOC, ISC, and $P_{max}$ are highest in the case with no filter is that there is no obstruction to the white light reaching the solar cell, and all wavelengths of visible light are present, leading to higher energy transmission to the electrons and facilitating their migration and lifetime.

		\item Based on this understanding, the filters can be ranked in terms of their maximum power output as follows: No filter > Green > Yellow > Pink > Red. However, the actual results from the graphs show a slightly different ranking: No filter > Yellow > Pink > Red > Green. This can be attributed to the fact that the green filter was very dark and blocking most of the light, which affected its overall performance. Also, in sunlight, the differences between the filters were less pronounced due to the higher intensity of the falling light.
		
		\item We can clearly see that for Pink, Red and Green, the theoretical ordering is maintained indeed in incandescent light, but the green filter was very dark and it was blocking most of the light from the spectrum. Hence it's overall performance suffered. Also in case of sunlight, we see that the performances of these filters are almost similar. The difference is only clear in low intensity indoor lighting. This is because, in sunlight, the intensity of the falling light is much higher than. Hence the effect of the filters is not as prominent as in indoor lighting.

		\item The lower power output of higher frequency radiation in a solar cell could be due to scattering, which makes collisions between higher wavelength photons more probable than lower ones, resulting in more scattering instead of absorption. This could explain why green light performed poorly and why yellow was ranked below pink. Other factors such as imperfections in the filter or misplacement of the solar cell may also have contributed.
		
		\item Another reason could be the cloudy conditions that came up after taking the reading for green filter in sunlight that significantly lowered the output for the other filters. Hence the same results as the lamp case weren't reproduced.
	\end{enumerate}
