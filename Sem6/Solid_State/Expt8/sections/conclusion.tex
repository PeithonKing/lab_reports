\section{Conclusion}
	Based on the collected data, it can be concluded that the capacitance of solar cells is inversely proportional to the light intensity. This implies that the capacitance of a solar cell will be lower in the presence of higher light intensities such as during sunny days. The reason for this behavior can be attributed to the dependence of capacitance on the doping density of the p-n junction. Hence, it is possible to increase capacitance by doping the p-n junction more. Another way to enhance capacitance is by using a material with a high dielectric constant. As a result, it can be predicted that larger solar cells will have higher capacitance values since capacitance is dependent on the area.

	To further explain, doping is a process of introducing impurities into the semiconductor material to increase the number of free charge carriers. This increases the capacitance by decreasing the depletion region width of the p-n junction. Similarly, the use of materials with a high dielectric constant results in higher capacitance since the dielectric constant directly affects the amount of electric charge that can be stored per unit area. The area-dependent nature of capacitance can be understood from the fact that it is proportional to the area of the p-n junction. Therefore, larger solar cells will have a higher capacitance value compared to smaller ones.