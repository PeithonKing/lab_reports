\begin{abstract}
	We present an inexpensive experiment to measure the gravitational acceleration (g) using a Seelab microcontroller-based system. The experiment uses a basic harmonic oscillating pendulum and an LED-photo transistor setup to determine the time period of the oscillation. Readings are automatically taken, and the constants are computed using Python code.
	\begin{description}
		\item[Lorem] The first part of the experiment is conducted in a lab frame, where the pendulum oscillates under the influence of gravity. The time period of the pendulum is measured using the LED-photo transistor assembly, and the value of g is calculated.
		\item[Ipsum Dolor] In the second part of the experiment, the pendulum experiment is repeated in an elevator frame to determine the acceleration of the lift. By comparing the period of the pendulum in the lab frame to the period in the elevator frame, the acceleration due to the lift's motion is calculated. This extension provides a way to measure gravity in a non-inertial reference frame.
	\end{description}
\end{abstract}
