\section{Observation and Calculation}

	\subsection{In Lab Frame}
		We took the oscillation data for 30 seconds. Then using the above mentioned \hyperref[code:time_period]{\texttt{time\_between\_peaks()} function}, and multiplying it by 500 (explained earlier) we get the average Time Period of oscillation in seconds. We then calculated the acceleration of the lift using the \hyperref[eq:1]{Equation 1}. 

		From the code, we got the Time Period of the pendulum as: \textbf{0.5185s}. Thus from the \hyperref[eq:1]{Equation 1}, we got the acceleration of the lift as: \textbf{9.79m/s$^2$}.

	\subsection{In Lift Frame}
		In lift, we observed that the SPS lift takes around 15 seconds to go from the ground floor to the 4th floor. So, we took reading for 20-25 seconds while the lift was moving. Then we found out the time periods between the peaks and added every odd and even distances. Thus we got an array of time periods of the pendulum at different times. We got \hyperref[th:4]{Figure 4} by plotting the time periods with time. next we labelled the peaks in the graph manually.

		We take the extreme points in those peaks and intend to find the maximum acceleration and decelaration of the lift.

		From the graphs\footnote{Note that in \hyperref[th:4]{Figure 4} for acceleration time period is less and for decelaration time period is more} we found that, \textbf{Average Maximum Acceleration Time Period} is $0.492s$ and \textbf{Average Maximum Deceleration Time Period:} $0.544s$.
		% \vspace{3mm}

		The length of the pendulum rod is measured to be $10cm$
		% \vspace{3mm}
		
		Now using \hyperref[eq:1]{Equation 1} we get, while the lift is accelerating, the acceleration of the lift is \textbf{10.873 m/s$^2$} and while the lift is decelerating, the acceleration of the lift is \textbf{8.893m/s$^2$}.
		% \vspace{3mm}

		Now, the original acceleration of the lift is \textbf{9.79m/s$^2$}. So, the maximum acceleration of the lift with respect to the earth is \textbf{1.083m/s$^2$}. Similarly, the maximum deceleration of the lift with respect to the earth is \textbf{0.896m/s$^2$}.