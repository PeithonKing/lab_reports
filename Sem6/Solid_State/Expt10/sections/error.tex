\section{Error Analysis}
	\begin{itemize}
		\item \textbf{Alignment of Pendulum:} Even though the pendulum was oscillating on a knife edge, there could have been some misalignment in the pivot point which could have introduced errors in the measurement of the time period. To minimize this error, we made sure to align the pendulum carefully before taking measurements.
		\item \textbf{Peak Detection:} The pendulum rod was thick, so we had to take the falling curve (when the pendulum is going out) while detecting a peak. We should have taken the middle point in the peak to evade errors. Due to this, there could have been some error in the measurement of the time period of the pendulum.
		\item \textbf{Brief Lift Acceleration:} During the second part of the experiment, we conducted the same experiment inside a lift to find the acceleration of the lift. However, the lift accelerated and decelerated for a very small amount of time, and we could not take many data points during this short period. This limitation could have affected the accuracy of our measurements.
		\item \textbf{Lift Vibration and Jerk:} In the lift, there could also have been some vibration or jerk due to the lift's motion, which could have affected the motion of the pendulum and introduced error in the measurement of the time period.
		\item \textbf{Effect of Air Resistance:} Although the pendulum rod was small, there could still have been some effect of air resistance on the motion of the pendulum. This effect would be negligible but not completely negligible. To reduce this error, we made sure to keep the air around the pendulum as still as possible during the experiment.
	\end{itemize}