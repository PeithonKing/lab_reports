\section{Conclusion}
	In this experiment, we looked at how the dielectric constant and the capacitance of $BaTiO_3$, MLCC, and DCC, which decreases with frequency, vary with frequency and temperature. The dielectric constant lowers as the total polarisation decreases and the relaxation times of various polarisation mechanics result in slower alignment of polarisation vectors in the direction of the electric field as compared to the frequency of the electric field.
	
	We also investigate $BaTiO_3$'s phase transition. The dielectric constant of ($BaTiO_3$) is seen to grow up to the Curie temperature before decreasing after our theoretical prediction has been confirmed. This shows that barium titanate exhibits ferroelectric behaviour below the Curie temperature (where it exists in a tetragonal phase with a net dipole moment) and paraelectric behaviour above $T_c$ (where it exists in a cubic phase without a net dipole moment). We may infer that the Curie temperature is unaffected by the frequency of the applied electric field because it is the same for each frequency curve.
	
	We found that frequency had little effect on the variable resistance. Only the amplitude of the signal on the oscilloscope was impacted by adjusting the variable capacitor's capacitance. But since the sample's resistance is not necessary, we may infer that the balancing resistance is essentially independent of C.
	
	We may infer that the sample includes impurities since the measured curie temperature deviates from the predicted value.
 