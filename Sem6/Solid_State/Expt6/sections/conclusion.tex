\section{Conclusion}
	In this experiment, we studied the frequency and temperature variation of dielectric constant and capacitance of $BaTiO_3$, MLCC and DCC, which decreases with an increase in frequency. The dielectric constant decreases as the relaxation time of different polarization mechanics cause the slower alignment of polarisation vectors in the direction of the electric field as compared to the frequency of the electric field, thus decreasing the total polarisation.
	
	We also study the phase transition of $BaTiO_3$. We observe that  the dielectric constant of ($BaTiO_3$) increases up to the Curie temperature then decreases after it validates our theoretical prediction. This indicates that Barium Titanate behaves as a ferroelectric material(existing in a tetragonal phase with a net dipole moment) below the Curie temperature and as a paraelectric (existing in a cubic phase without a net dipole moment) above Tc. Since the Curie temperature is the same for each frequency curve, we can conclude that the Curie temperature is independent of the frequency of the applied electric field.
	
	As the observed curie temperature is different from the theoretical value, we can say that sample contains impurities leading to this increase.
	
	We noted that the variable resistance didn't change much with frequency  Changing the capacitance of the variable capacitor only affected the amplitude of the signal on the oscilloscope. However, since the resistance of the the sample is not required, we can conclude that the balancing resistance is more or less independent of C.
 