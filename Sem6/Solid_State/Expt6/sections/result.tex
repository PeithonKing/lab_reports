\section{Results and Discussion}
    \begin{itemize}
        \item \textbf{For frequency = 5 kHz:} slope $(\delta) = 1.1231$
        \item \textbf{For frequency = 15 kHz:} slope $(\delta) = 1.1689$
        \item \textbf{For frequency = 25 kHz:} slope $(\delta) = 1.68349$
        \item \textbf{For frequency = 35 kHz:} slope $(\delta) = 1.32536$
        \item Curie temperature $T_c = 140^\circ C$ From the graphs show the diffuseness parameter ($\delta$) for each frequency and the corresponding error.
        \item The optimal size of $BaTiO_3$ and good temperature management can reduce the huge error in curie temperature.
        \item Minimum deviation from the linear curve is shown in the graphs for the diffuseness parameters. The observation is accurate since the actual value exceeds the theoretically predicted range of 1 to 2.
        \item We can see that the capacitance/dielectric constant drops with an increase in the frequency of the external field since the plots of dielectric constant or capacitance for $BaTiO_3$, MLCC, and DCC corresponded pretty well with the theoretical expectations.
        \item We discover that the temperature dependency plot is comparable to the predicted plot, but the curie temperature is changed, and the reduction starts to happen at temperatures over $140^\circ C$. A fractured and thick sample of $BaTio_3$ and impurities may be to blame for this notable discrepancy in curie temperature. Additionally, the temperature at the time of reading could not be correct because the oven temperature was fast rising.
    \end{itemize}
 
    \subsection{Scope of Error:}
        \begin{itemize}
            \item Sample impurities that raise the curie temperature
            \item The variable resistor may saturate before the Curie temperature if a sample is taken with a high sample resistance, which will alter the predicted plot form.
            \item The sample not being properly in touch with the probes. and loose connections in any place of the circuit. improperly connected wires can increase the capacitance in the circuit.
            \item Readings beyond the lowest allowable voltage amplitude may result in variable resistance inaccuracy.
        \end{itemize}
 
 