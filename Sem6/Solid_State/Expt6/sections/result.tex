\section{Results and Discussion}
    \begin{itemize}
        \item Curie temperature $T_c = 140^\circ C$ From the graphs show the diffuseness parameter ($\delta$) for each frequency and the corresponding error.
        \item For frequency = 5 kHz: slope of the line $(\delta) = 1.1231$
        \item For frequency = 15 kHz: slope of the line $(\delta) = 1.1689$
        \item For frequency = 25 kHz: slope of the line $(\delta) = 1.68349$
        \item For frequency = 35 kHz: slope of the line $(\delta) = 1.32536$
        \item The plots of dielectric constant or capacitance for $BaTiO_3$, MLCC and DCC matched reasonably with the theoretical expectations and we can observe that the capacitance/dielectric constant decreases with an increase in the frequency of the external field.
        \item For the temperature dependence plot, we find that is plot is similar to the expected plot however, curie temperature is shifted, and  The decrease occurs after $140^\circ C$. Temperature. This significant difference in curie temperature can be due to a broken and thick sample of $BaTio_3$ and impurities. Also, as the oven temperature was increasing rapidly, the temperature at the reading time might not be accurate.
        \item The plots for the diffuseness parameters show minimum deviation from  the linear curve. The observed value is more than 1, where the theoretically expected is between 1 and 2; thus, the observation is correct.
        \item Error in curie temperature is large and can be minimized by controlling the temperature properly and optimum size of $BaTiO_3$.
    \end{itemize}
 
    \subsection{Scope of Error:}
        \begin{itemize}
            \item Impurities in the sample which increase the curie temperature.
            \item Improper contact of the sample with the probes.
            \item Loose connections of the cables with the oscilloscope and Schering bridge apparatus.
            \item Taking a sample with high sample resistance can cause the variable resistor to saturate before Curie temperature, resulting in a different plot shape than the expected one.
            \item Taking readings above the minimum possible voltage amplitude can give error in variable resistance.
        \end{itemize}
 
 