\section{Results}
\begin{itemize}
	\item From \hyperref[tab:malus_law]{Table 1}, \hyperref[fig:6]{Figure 6} and \hyperref[fig:5]{Figure 5}, we find that the intensity of the polarized light follows Malu's law, as $I$ is proportional to $\cos^2{\theta}$.
	\item From \hyperref[tab:half]{Table 2} and \hyperref[fig:7]{Figure 7}, we find that for the half wave plate kept at $\theta=46\degree$ angle, we obtain two maximas at $29\degree$ and $204\degree$ showing that the light beam has rotated by an angle of $90\degree \approx \;2\theta$.
	\item From \hyperref[tab:quarter]{Table 3}, \hyperref[fig:8]{Figure 8} and \hyperref[fig:9]{Figure 9}, we obtain a peanut shaped graph indicating an elliptically polarized light beam for the quarter wave plate angular position of $46\degree$. We can clearly see from \hyperref[fig:8]{Figure 8} that the intensity of the beam never goes to 0 and hence the beam is elliptically polarized.
	% \item The axes alignment of the plates can be determined from by coinciding the reflected rays from the plates.
\end{itemize}


\section{Error and Discussion}

This experiment doesn't have any such calculations. So, there is no such error which should be calculated and stated. But, there are a lot of errors which can be discussed.

\begin{itemize}
	\item \textbf{Error in Equipments:} The lazer intensity hardly stabilizes ever and the intensity of the beam keeps on changing. So, the readings keep on changing frequently. \textbf{I suspect the problem is with the photodiode and not with the lazer}. We went to the lab 2 days and came back without taking any reading. then on the third day we were fortunate to get a good lazer and got good reading.
	\item The lab manual asks us whether the lazer gives out a polarized light or not. To answer that question, I would say, that it gives \textbf{polarized light}. This is because, with only the polariser present between the photodiode and the lazer, the intensity changes with the angle of the polariser.
	\item In \hyperref[fig:6]{Figure 6} and \hyperref[fig:8]{Figure 8}, there is not very significant error in the data. But, in \hyperref[fig:7]{Figure 7}, there is a lot of error in the data. This is because, the lazer intensity keeps on changing and hence the readings keep on changing. So, the data is not very accurate.
\end{itemize}